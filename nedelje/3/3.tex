\section{Median String}

Funkcija ima dva parametra, \texttt{dna} - niz niski koje čine nukleotidnu sekvencu, i \texttt{k} - željena dužina . Povratna vrednost je niska \texttt{median}.

Tražimo nisku dužine \texttt{k} koja je najmanje udaljena od svih iz \texttt{dna}. Uzimamo u obzir svih $4^k$ kombinacija i proveravamo koja najmanje udaljenda od svih. Za određivanje niske koristi se funkcija \texttt{number\_to\_pattern} (\ref{numberToPattern}). Zatim, za dobijenu nisku određujemo ukupno rastojanje od svih sekvenci iz \texttt{dna} (funkcijom \texttt{d}). Potrebno je još proveriti da li je to rastojanje trenutni minimum i, ako jeste, ažurirati minimalno rastojanje i nisku \texttt{meadian}.

\lstinputlisting[language=Python, frame=single]{nedelje/3/kodovi/1.MedianString.py}

\subsection{D}

Funkcija ima dva parametra, \texttt{pattern} - kandidat za median, i \texttt{dna} - niz niski koje čine nukleotidnu sekvencu. Povratna vrednost je suma Hamingovih rastojanja između niske \texttt{pattern} i svih sekvenci iz \texttt{dna}.

Prolazimo jednu po jednu sekvencu iz \texttt{dna} i za svaku određujemo najmanje Hamingovo rastojanje u odnosu na \texttt{pattern}. Ideja je da u toj sekvenci tražimo podsekvencu, dužine niske \texttt{pattern} (u kodu, to je vrednost \texttt{k}), koja se najbolje poklapa sa niskom \texttt{pattern} odnosno, koja ima najmanje hamingovo rastojanje (\ref{hammingDistance}) u odnosu na nisku \texttt{pattern}. Nakon što je određeno najmanje rastojanje za jednu sekvencu, ono se dodaje na ukupan zbir rastojanja, što je povratna vrednost ove funkcije.

\lstinputlisting[language=Python, frame=single]{nedelje/3/kodovi/1.1.D.py}



\subsubsection{Test primer}
\noindent\texttt{dna} = [ \\
'GTAGATGTCATTAGCATGCAC', \\
'CCTAGCCACTCTGCCATGTCG', \\
'AACTCGTGCATTCTACGACTG', \\
'AAACTTTCCGGATCTTCATAC', \\
'CTACATCATCGAAGGCTACGC' \\
]

\noindent \texttt{k} = 4
\\\texttt{median} =


\section{Greedy Motif Search}

Funkcija ima tri parametra, \texttt{dna} - niz sekvenci, \texttt{k} - dužina motiva, i \texttt{t} - broj sekvenci u \texttt{dna}. Povratna vrednost je skup najboljih motiva.

Inicijalno, \texttt{best\_motifs} sadrži prefikse svih niski iz \texttt{dna} dužine \texttt{k}. Pored najboljih motiva, pamtimo i trenutno nabolji (najmanji) skor, koji se računa koršćenjem funkcije \texttt{score}. 

Polazimo od prve sekvence u nizu \texttt{dna} iz koje ćemo izdvajati podniske dužine \texttt{k}. Čuvamo skup motiva, a prvi motiv biće podniska dužine \texttt{k} u tekućoj iteraciji. Dakle, indeksiramo prvu nisku i u svakoj iteraciji izdvajamo podniske dužine \texttt{k}. Pored toga, u svakoj iteraciji iz preostalih \texttt{t} - 1 sekvenci izdvajamo podniske dužine \texttt{k} koje su najverovatnije (to određuje funkcija \texttt{most\_probable\_kmer}) u odnosu na matricu \texttt{profile}. Određujemo profil na osnovu motiva iz prethodnih iteracija (a dobijamo ga kao povratnu vrednost funkcije \texttt{profile}), a onda i najverovatniju podnisku. Najverovatniju podnisku dodajemo u skup motiva i on se koristi u narednoj iteraciji za pronalaženje sledećeg motiva. Kada su određeni svi motivi u tekućoj iteraciji, računa se trenutni skor i, ukoliko je bolji od trenutnog najboljeg, ažuriramo najbolji skor i motive.


\lstinputlisting[language=Python, frame=single]{nedelje/3/kodovi/2.GreedyMotifSearch.py}


\subsection{Score}
\label{score}

Funkcija ima dva parametra, \texttt{motifs} - skup motiva, i \texttt{k} - dužina svakog od motiva. Povratna vrednost je ukupan skor. Da se podsetimo, skor predstavlja ukupan broj nepopularnih nukleotida po kolonama.

Promenljiva \texttt{t} pamti broj motiva, a povratna vrednost biće smeštena u promenljivu \texttt{total\_score}. Dakle, krećemo se po kolonama, tako da će granica za izvršavanje petlje biti vrednost \texttt{k}, što je dužina svakog od motiva. Za svaku kolonu pamtimo koliko se koji nukleotid pojavio u nizu \texttt{counts} dimenzije 4. Zatim prolazimo redom karaktere motiva iz odgovarajuće kolone i ažuriramo niz. Koristimo funkciju \texttt{symbol\_to\_number} (\ref{symbolToNumber}) za dobijanje indeksa, a potom sledi jednostavna inkrementacija elementa niza na odgovarajućoj poziciji. Kada su obrađeni svi motivi, potrebno je odrediti indeks maksimuma. Na kraju, uvećati \texttt{total\_score} za razliku ukupnog broja nukleotida i broj pojavljivanja najpopularnijeg nukleotida.

\lstinputlisting[language=Python, frame=single]{nedelje/3/kodovi/2.4.Score.py}


\subsection{Profile From Motifs}
\label{profileFromMotifs}

Funkcija ima dva parametra, \texttt{motifs} - skup motiva, i \texttt{k} - dužina svakog motiva. Povratna vrednost funkcije je matrica profila dimenzija $4 \times k$. Podsetimo se da vrednosti u matrici profila predstavljaju verovatnoću da se pojavi odgovarajući nukleotid na odgovarajućem mestu (ako bacamo četvorostranu pristrasnu kockicu, ove vrednosti bi bile verovatnoće da padne odgovarajući nukleotid). U ovoj implementaciji primenjeno je i Laplasovo pravilo, kako bi se izbegla vreovatnoća jednaka nuli.


Zbog Laplasovog pravila, matrica profila inicijalno je popunjena jedinicama. U prvom prolazu dvostruke petlje, koja se prvo kreće po kolonama a onda po vrstama, određujemo indeks nukleotida, a onda uvećamo odgovarajuće polje u matrici profila. U drugom prolazu dvostruke petlje, delimo svaki element matrice brojem motiva uvećanog za 2.


\lstinputlisting[language=Python, frame=single]{nedelje/3/kodovi/2.1.ProfileFromMotifs.py}


\subsection{Most Probable Kmer}
\label{mostProbableKmer}

Funkcija ima tri parametra, \texttt{dna\_string}, \texttt{profile} i \texttt{k}. Povratna vrednost funkcije je najverovatnija podniska niske \texttt{dna\_string} dužine \texttt{k} u odnosu na amtricu profila.

Iniciijalno, \texttt{best\_kmer} je prazna niska, a \texttt{best\_probability} je negativna vrednost. Prolazimo sve podniske dužine \texttt{k} i za svaku računamo verovatnoću (funkcijom \texttt{probability}). Ukoliko je ta verovatnoća veća od trenutno najbolje, ažuriramo najbolju podnisku i verovatnoću.

\lstinputlisting[language=Python, frame=single]{nedelje/3/kodovi/2.3.MostProbableKmer.py}


\subsubsection{Probability}
\label{probability}

Funkcija ima dva parametra, \texttt{pattern} i \texttt{profile}. Funkcija vraća verovatnoću pojave \texttt{pattern} sekvence u odnosu na zadati profil. Ukupna verovanoća dobija se kao proizvod odgovarajućih vrednosti iz matrice profil.

Inicijalno, verovatnoća je jednaka 1. Za svaki karakter iz sekvence \texttt{pattern}, određujemo njegov indeks funkcijom \texttt{symbol\_to\_number} (\ref{symbolToNumber}). Zatim, trenutnu verovatnoću množimo vrednošću iz matrice profil koja odgovara datom karakteru i poziciji na kojoj se nalazi u niski.

\lstinputlisting[language=Python, frame=single]{nedelje/3/kodovi/2.2.Probability.py}


\subsubsection{Test primer}
\noindent\texttt{dna} = [ \\
'GTAGATGTCATTAGCATGCAC', \\
'CCTAGCCACTCTGCCATGTCG', \\
'AACTCGTGCATTCTACGACTG', \\
'AAACTTTCCGGATCTTCATAC', \\
'CTACATCATCGAAGGCTACGC' \\
]

\noindent \texttt{k} = 4
\\\texttt{t} = len(dna)
\\\texttt{best\_motifs} =


\section{Randomized Motif Search}

Funkcija ima tri parametra, \texttt{dna}, \texttt{k} i \texttt{t}. Povratna vrednost je skup najboljih motiva.

Polazimo od slučajno odabranih motiva (dobijamo ih funkcijom \texttt{random\_k\_mers}). Pamtimo trenutno najbolje motive (na početku su to ovi slučajno odabrani) i njihov skor (koji računamo korišćenjem funkcije \texttt{score}, \ref{score}).

Vrtimo petlju dok se skor popravlja. Prvo, formiramo profil od tekućih motiva (funkcijom \texttt{profile\_from\_motifs}, \ref{profileFromMotifs}), a onda obrnuto, određujemo motive od dobijenog profila (funkcijom \texttt{motifs\_from\_profile}). Računamo skor novih motiva, i ako je skor manji od najboljeg, ažuriramo najbolje motive i njihov skor. Ukoliko je novi skor lošiji, tada prekidamo petlju i vraćamo najbolje motive.



\lstinputlisting[language=Python, frame=single]{nedelje/3/kodovi/3.RandomizedMotifSearch.py}

\subsection{Random Kmers}
\label{randomKmers}

Funkcija ima tri parametra, \texttt{dna}, \texttt{k} i \texttt{t}. Povratna vrednost je skup slučajno odabranih motiva. 

Skup motiva na početku je prazan. Za svaku sekvencu iz \texttt{dna} generišemo jedan sluačajan broj koji predstavlja poziciju od koje počinje motiv, a biće dužine \texttt{k}. Odabranu podsekvencu dodajemo u skup motiva.

\lstinputlisting[language=Python, frame=single]{nedelje/3/kodovi/3.2.RandomKmers.py}

\subsection{Motifs From Profile}
\label{motifsFromProfile}

Funkcija ima dva parametra, \texttt{profile} i \texttt{dna}. Povratna vrednost je skup motiva.

Polazi se od praznog skupa motiva. Za svaku sekvencu iz \texttt{dna} određujemo najverovatniju podnisku (korišćenjem funkcije \texttt{most\_probable\_kmer}, \ref{mostProbableKmer}) i dodajemo je u skup motiva.

\lstinputlisting[language=Python, frame=single]{nedelje/3/kodovi/3.1.MotifsFromProfile.py}

\subsubsection{Test primer}
\noindent\texttt{dna} = [ \\
'GTAGATGTCATTAGCATGCAC', \\
'CCTAGCCACTCTGCCATGTCG', \\
'AACTCGTGCATTCTACGACTG', \\
'AAACTTTCCGGATCTTCATAC', \\
'CTACATCATCGAAGGCTACGC' \\
]

\noindent \texttt{k} = 4
\\\texttt{t} = len(dna)
\\\texttt{best\_motifs} =

\section{Gibbs Sampler}

Funkcija ima četiri parametra, \texttt{dna}, \texttt{k}, \texttt{t} i \texttt{N} - broj iteracija. Povratna vrednost je skup najboljih motiva.

Polazimo od slučajno odabranih motiva (dobijamo ih funkcijom \texttt{random\_k\_mers}). Pamtimo trenutno najbolje motive (na početku su to ovi slučajno odabrani) i njihov skor (koji računamo korišćenjem funkcije \texttt{score}, \ref{score}).

Funkcija se izvršava u \texttt{N} iteracija, a u svakoj prvo biramo slučajan broj \texttt{i} iz skupa $[0, t)$ (to je slučajno odabrani motiv koji brišemo). Koristimo pomoćnu promenljivu \texttt{selected\_motifs} u koju kopiramo elemente iz \texttt{motifs}, a onda brišemo onaj na poziciji \texttt{i}, koja je slučano odabrana. Zatim, pravimo matricu profila (\texttt{profile\_from\_motifs}, \ref{profileFromMotifs}). Na osnovu dobijenog profila, određujemo najverovatniji motiv (funkcijom \texttt{most\_probable\_kmer}, \ref{mostProbableKmer}) i smeštamo u \texttt{motifs} na prehodno odabranu poziciju \texttt{i}, a brišemo ceo \texttt{selected\_motifs}.

Nakon toga, određuje se skor novog skupa motiva i, ukoliko je manji, ažuriramo najbolje motive i njihov skor.
  

\lstinputlisting[language=Python, frame=single]{nedelje/3/kodovi/4.GibbsSampler.py}



\subsubsection{Test primer}
\noindent\texttt{dna} = [ \\
'GTAGATGTCATTAGCATGCAC', \\
'CCTAGCCACTCTGCCATGTCG', \\
'AACTCGTGCATTCTACGACTG', \\
'AAACTTTCCGGATCTTCATAC', \\
'CTACATCATCGAAGGCTACGC' \\
]

\noindent \texttt{k} = 4
\\\texttt{t} = len(dna)
\\\texttt{N} = 500
\\\texttt{best\_motifs} =