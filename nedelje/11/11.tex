\section{HMM}

Funkcija ima dva argumenta, listu niski \texttt{strings\_pos} i \texttt{string\_neg} (pozitivno i negativno, kao kod primera sa bacanjem pristrasnog, što je označavalo negativne niske, i fer novčića, što je označavalo pozitivne niske). Povratna vrednost je HMM dijagram.

HMM na početku je prazna mapa. Ključevi u mapi su niske sastavljene od karaktera i znaka '+' ili '-', a vrednosti su mape. Svaka od tih mapa sadrži tri ključa: \texttt{'state'} - 1 -pozitivni, 0 - negativni; \texttt{'nucleotide'} i \texttt{'transitions'} - mapa prelazaka (ključevi su karakteri, a vrednosti su brojevi).

Prvo prolazimo listu pozitivnih stringova, element po element, a onda i negativne. \footnote{Pošto je posao za pozitivne i negativne većinom isti, opisaćemo obe petlje zajedno, a delove koji se razlikuju opisujemo u zagradama za negativne niske.} Svaku nisku prolazimo pomoću indeksa \texttt{i}. Pamtimo dve promenljive, \texttt{c\_prev\_state}, koja dobija vrednost karaktera na prethodnoj poziciji na koji nadovezujemo znak '+' (odnosno, '-' za negativne niske), i \texttt{c\_curr} koja označava trenutni karakter u niski.

Ukoliko se prethodni karakter ne nalazi u \texttt{HMM} dodajemo ga sa praznom mapom. Zatim mu inicijalizujemo \texttt{state} na 1 (odnosno, 0 za negativne). Vrednost za ključ \texttt{nucleotide} je prethodni karakter niske, a \texttt{transition} dobija praznu mapu.

Ukoliko se trenutni karakter ne nalazi u tranzicijama \texttt{c\_prev\_state}, dodajemo ga i inicijalizujemo mu vrednost na 0. U svakom slučaju ćemo uvećati tu vrednost za 1.

Sada prolazimo sve ključeve u \texttt{HMM}. Računamo sumu vrednosti svih tranzicija. Zatim, vrednost svake tranzicije delimo tom sumom i množimo sa 0.98. Pre prelaska na sledeću iteraciju, dodajemo još jedan element u mapu, a to je ('0', 0.02).



\lstinputlisting[language=Python, frame=single]{nedelje/11/kodovi/1.HMM.py}

\section{Viterbi}


Funkcija ima dva parametra, \texttt{HMM} i \texttt{string}. Povratna vrednost je putanja \texttt{path}.

Putanja je na početku prazna niska, a pored toga održavamo listu, na početku praznih, mapa, po jednu za svaki karakter niske. 

Pomoću indeksa \texttt{i} redom prolazimo karaktere niske. Izdvajamo trenutni karakter u promenljivu \texttt{nucleotide}. U svakoj iteraciji, osim prve, u putanju dodajemo najveće stanje prelaska. Promenljive \texttt{max\_transition\_prob} i \texttt{max\_transition\_state} postavljamo na -1, odnosno, na praznu nisku.

Za svako stanje u \texttt{HMM} proveravamo da li je njegov nukleotid jednak trenutnom nukleotidu. Ukoliko nije, u \texttt{matrix} postavljamo 0 za \texttt{i}-ti element i trenutno stanje. Ukoliko jeste, u prvoj iteraciji postavljamo tu vrednost na 0.125, a u svim ostalim iteracijama radimo sledeće.

Prvo, postavljamo promenljive \texttt{max\_prev} na -1 i \texttt{max\_prev\_state} na praznu nisku. Za svako prethodno stanje u \texttt{matrix} prethodnog elementa proveravamo da li se prošlo i trenutno stanje poklapaju. Ako da, postavljamo \texttt{state\_change\_prob} na 0.99, inače na 0.01. Zatim, ako se nukleotid trenutnog stanja nalazi u tranzicijama prethodnog stanja onda \texttt{transition\_prob} dobija vrednost tranzicije prethodnog stanja za nukleotid trenutnog stanja, a inače 0.

Još uvek obrađujemo treći slučaj, a sledeći korak jeste da odredimo vrednost trenutnog stanja. Promenljiva \texttt{curr\_state} dobija vrednost proizvoda  vrednosti prethodnog stanja \texttt{i-1}-og elementa liste \texttt{matrix} sa \texttt{state\_change\_prob} i \texttt{transition\_prob}. Ukoliko je trenutno stanje veće od \texttt{max\_prev}, ažuriramo maksimume, tako da \texttt{max\_prev} dobije vrednost \texttt{curr\_state}, a \texttt{max\_prev\_state} dobije vrednost \texttt{prev\_state}.

Vrednost za ključ \texttt{state} \texttt{i}-tog elementa liste dobija vrednost \texttt{max\_prev} i ostaje još da uporedimo \texttt{max\_prev} i \texttt{max\_transition\_prob}. Ukoliko je veće ažuriramo maksimume tako da \texttt{max\_transition\_prob} dobije vrednost \texttt{max\_prev}, a \texttt{max\_transition\_state} dobije vrednost \texttt{max\_prev\_state}.

Sada su obrađeni svi karakteri niske i treba da odredimo završno stanje. Promenljiva \texttt{max\_end} dobija vrednost -1, a \texttt{max\_end\_state} je prazna niska. Prolazimo sva stanja poslednjeg elementa liste i tražimo ono sa najvećom vrednošću.

Na putanju nadovezujemo \texttt{max\_end\_state} i vraćamo putanju.


\lstinputlisting[language=Python, frame=single]{nedelje/11/kodovi/2.Viterbi.py}


\subsection{Test primer}

\noindent\texttt{strings\_pos} = [
\\\indent 'ACACAGACGCACA',
\\\indent 'CACATAGACAGGCATACACA', \\\indent 'AAATACAGTATCTTTGCACTCCCGGAGTGCGG'\\]
\\\texttt{strings\_neg} = [\\\indent 'CGAGCGTGTGAGTGAGAGATGAG', \\\indent 'GTGGAACAGTAGGTAGGAGAGTG', \\\indent 'AAATACAGTATCTTTGCACTCCCGGAGTGCGG']
\\\texttt{model = HMM(strings\_pos, strings\_neg)}
\\\texttt{path} = 