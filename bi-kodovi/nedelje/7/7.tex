\section{Affine Gap Alignment}

Funkcija ima dva parametra, niske \texttt{string\_1} i \texttt{string\_2}. Povratna vrednost je vrednost najvećeg poravnanja.

Održavamo tri matrice, \texttt{upper}, \texttt{lower} i \texttt{middle}. Svaka je dimenzija (|string\_1|+1) $\times$ (|string\_2|+1) i svakoj su vrednosti inicijalizovane na $-\infty$. Jedino prvi element srednje matrice ima vrednost 0. 


U dvostrukoj petlji računamo vrednosti za svaki element ovih matrica kao maksimume tri broja. Prelazak sa jednog nivoa na drugi, odnosno, iz jedne matrice u drugu, kažnjavamo sa -10 (to je pojava nove rupe - gap). Svaki sledeći karakter u rupi kažnjavamo sa -0.5. 

Za gornju matricu to su: element središnje matrice iz prethodne kolone umanjen za 10 i 0.5, element gornje matrice iz prethodne kolone umanjen za 0.5 i element donje matrice iz prethodne kolone umanjen za 10 i 0.5. 

Za donju matricu to su: element središnje matrice iz prethodne vrste umanjen za 10 i 0.5, element gornje matrice iz prethodne vrste umanjen za 10 i 0.5 i element donje matrice umanjen za 0.5. 

Za središnju matricu to su: element središnje matrice iz prethodne kolone i prethodne vrste uvećan za povratnu vrednost funkcija \texttt{match} za karaktere niski na prethodnim pozicijama (\ref{match2}), element gornje matrice i element donje matrice. 

Povratna vrednost je maksimum poslednjih elemenata ove tri matrice.

\lstinputlisting[language=Python, frame=single]{nedelje/7/kodovi/1.AffineGapAlignment.py}

\subsection{Match}
\label{match2}

Funkcija ima dva parametra, karakteri \texttt{s1} i \texttt{s2}. Ukoliko su karakteri jednaki, vraćamo vrednost 1, u suprotnom -4.

\lstinputlisting[language=Python, frame=single]{nedelje/7/kodovi/1.1.Match.py}

\subsubsection{Test primer}

\noindent \texttt{string\_1} = 'ACGTGCTCG'
\\\texttt{string\_2} = 'AATGCTCT'
\\\texttt{alignment} = -12.5
