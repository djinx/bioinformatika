\section{Maximal Non Branching Path}

Funkcija ima jedan parametar, \texttt{G} - de Bruijn graf. Funkcija vraća maksimalne nerazgranate putanje u grafu.

Putanja je inicijalno prazna lista, a dodatno održavamo mapu posećenih čvorova, koja je inicijalno prazna.

Za svaki čvor \texttt{v} u grafu \texttt{G} računamo ulazni i izlazni stepen (funkcijom \texttt{degree}, \ref{degree}). Ukoliko je jedan od ta dva različit od 1, obeležavamo da je čvor posećen. Dodatno, ukoliko je izlazni stepen veći od 0, želimo da obiđemo sve susede. Za svaki čvor \texttt{w} iz liste suseda čvora \texttt{v}, pravimo putanju koja na početku sadrži samo granu (v, w). Zatim, obeležimo da je čvor \texttt{w} posećen i računamo njegov ulazni i izlazni stepen. 

Pošto nam je potreban nerazgranati put, pratimo čvorove koji imaju tačno jednu ulaznu i jednu izlaznu granu. Tako, dokle god su i ulazni i izlazni stepeni čvora \texttt{w} jednaki 1, određujemo čvor \texttt{u}, čvor koji se nalazi na drugom kraju izlazne grane iz čvora \texttt{w}. Tu granu treba dodati u trenutni put. Zatim, čvor \texttt{w} dobija vrednost čvora \texttt{u} i potrebno je ponovo označiti čvor \texttt{w} kao posećen i ažurirate stepene. Kada nema više čvorova koji ispunjavaju uslov, u listu putanja dodajemo trenutni put.

Još jednom prolazimo sve čvorove grafa \texttt{G}, i za svaki čvor \texttt{v} koji nije posećen, odredićemo izolovani ciklus (funkcijom \texttt{isloated\_cycle}, \ref{isolatedCycle}). Ukoliko takav ciklus postoji, dodajemo ga u listu putanja.



\lstinputlisting[language=Python, frame=single]{nedelje/4/kodovi/1.MaximalNonBranchingPath.py}


\subsection{Degree}
\label{degree}

Funkcija ima dva parametra, \texttt{G} - de Bruijn graf, i \texttt{v} - čvor čiji ulazni i izlazni stepen treba odrediti. Povratna vrednost je par ulazni i izlazni stepen čvora \texttt{v}.

Pošto je graf predstavljen kao mapa, koja preslikava čvorove u listu čvorova sa kojim je povezan, izlazni stepen biće dužina liste čvora \texttt{v} u mapi \texttt{G}. Ulazni stepen je broj čvorova koji u svojoj listi sadrže čvor \texttt{v}. 

\lstinputlisting[language=Python, frame=single]{nedelje/4/kodovi/1.2.Degree.py}


\subsection{Isolated Cycle}
\label{isolatedCycle}

Funkcija ima dva parametra, \texttt{G} - de Bruijn graf, i \texttt{v} - izolovani čvor za koji želimo da nađemo ciklus. Povratna vrednost je ciklus, ako postoji, a u suprotnom \texttt{None}.

Ciklus je inicijalno prazna lista grana (odnoso, parova ulaznih i izlaznih čvora). Pošto tražimo nerazgranate puteve, i ovde je bitno da svi čvorovi imaju ulazne i izlazne stepene jednake 1. Tako da, prvi korak je određivanje stepena čvora \texttt{v} (funkcijom \texttt{degree}, \ref{degree}). Petlja se vrti dok su ulazni i izlazni stepen jednaki 1. Biramo čvor \texttt{u} koji se nalazi na drugom kraju jedine izlazne grane čvora \texttt{v} i dodajemo tu granu u ciklus. 

Ukoliko su ulazni čvor prve grane i izlazni čvor poslednje grane ciklusa jednaki, znači da smo napravili ciklus i, pritom, obišli sve čvorove u toj komponenti povezanosti grafa \texttt{G}, pa vraćamo ciklus. U suprotnom, čvor \texttt{v} dobija vrednost čvora \texttt{u}. Ponovo računamo stepen čvora \texttt{v}. Ukoliko se naiđe na čvor koji ne ispunjava uslove petlje, vratiti \texttt{None}.

\lstinputlisting[language=Python, frame=single]{nedelje/4/kodovi/1.1.IsolatedCycle.py}

U nastavku su funkcije neophodne da bismo došli do reprezentacije niske u obliku de Bruijn grafa.


\subsection{String To Kmers}
\label{stringToKmers}

Funkcija ima dva parametra, \texttt{dna\_string} - nisku koju želimo da rasparčamo da delove, i \texttt{k} - dužina delova.

Polazimo od prazne liste k-grama. Polazeći od pozicije 0, uzimamo podniske dužine \texttt{k} i smeštamo u listu.  


\lstinputlisting[language=Python, frame=single]{nedelje/4/kodovi/1.4.StringToKmers.py}


\subsection{De Bruijn}
\label{deBruijn}
	
Funkcija ima jedan parametar, \texttt{kmers} - lista k-grama. Povratna vrednost je graf predstavljen mapom. 

Prolazimo listu, element po element. Za svaki k-gram izdvajamo prefiks \texttt{u}, bez poslednjeg karaktera i sufiks \texttt{v}, bez prvog karaktera. Ukoliko se \texttt{u} nalazi u grafu \texttt{G}, a čvor \texttt{v} nije u njegovoj listi, dodajemo čvor \texttt{v} u listu čvora \texttt{u}. Ako se \texttt{u} ne nalazi u grafu, onda ga dodajemo u graf, a lista inicijlano sadrži samo \texttt{v}. Ako čvor \texttt{v} nije u grafu, dodajemo ga sa praznom listom.


\lstinputlisting[language=Python, frame=single]{nedelje/4/kodovi/1.3.DeBruijnGraph.py}



\subsubsection{Test primer}

\noindent \texttt{dna\_string} = "AATCGTGACCTCAACT"
\\\texttt{k} = 3                   
\\\texttt{k\_mers} = string\_to\_k\_mers(dna\_string, k)
\\\texttt{g} = debruijn\_graph\_from\_k\_mers(k\_mers)
\\\texttt{paths} = 


\section{All Euler Cycles}

Funkcija prima jedan parametra, graf \texttt{G}. Povratna vrednost je lista ciklusa.

Lista ciklusa na početku je prazna. Koristimo pomoćnu promenljivu, \texttt{all\_graphs}, koja čuva kopiju grafa \texttt{G} u strukturi \textit{deque}. Petlja se vrti dok ima grafova u \texttt{all\_graphs}, odnosno, dok je njegova dužina veća od nula. 

Izdvajamo jedan graf, \texttt{G\_p} pozivom funkcije \texttt{popleft()}. U promenljivu \texttt{v\_p} želimo da smestimo čvor sa ulaznim stepenom većim od jedan u grafu \texttt{G\_p}. Inicijalizujemo ga sa \texttt{None} i onda pokušavamo da ga pronađemo.

Prolazimo sve čvorove grafa \texttt{G\_p} i računamo njihov ulazni i izlazni stepen (funkcija \texttt{degree}, \ref{degree}). Prvo čvor na koji naiđemo, sa ulaznim stepenom većim od 1, dodeljujemo promenljivoj \texttt{v\_p} i prekidamo potragu.

Ukoliko smo našli takav čvor, odnosno, ukoliko njegova vrednost nije jednaka \texttt{None}, želimo da napravimo jednostavniji $(u, v, w)$ bajpas graf. Da se podsetimo, potrebno je da iz grafa uklonimo grane $(u, v)$ i $(v, w)$ i da dodamo novi čvor $x$ sa granama $(u, x)$ i $(x, v)$.

Prolazimo sve čvorove \texttt{u} od kojih postoje grane ka čvoru \texttt{v\_p} u grafu \texttt{G\_p} (funkcija \texttt{incoming}, \ref{incoming}), i sve čvorove \texttt{w} do kojih postoji grana od čvora \texttt{v\_p} u grafu \texttt{G\_p} (funkcija \texttt{outgoing}, \ref{outgoing}). Zatim, pravimo bajpas graf (funkcijom \texttt{bypass}, \ref{bypass}). Ukoliko je novi graf povezan (što možemo proveriti funkcijom \texttt{is\_connected}, \ref{isConnected}), dodajemo njegovu kopiju u \texttt{all\_graphs}.

Ukoliko nismo našli odgovarajući čvor \texttt{v\_p}, onda prolazimo čvorove \texttt{k} grafa \texttt{G\_p} i određujemo izolovani ciklus iz svakog (funkcija \texttt{isloated\_cycle}, \ref{isolatedCycle}). Ukoliko takav ciklus postoji, želimo da napravimo nisku od ciklusa (funkcijom \texttt{create\_string\_from\_path}, \ref{createStringFromPath}) i da je dodamo u listu ciklusa, ukoliko se već ne nalazi tamo.


\lstinputlisting[language=Python, frame=single]{nedelje/4/kodovi/2.AllEulerCycles.py}

\subsection{Incoming}
\label{incoming}

Funkcija ima dva parametra, graf \texttt{G} i čvor \texttt{v}. Funkcija vraća listu čvorova grafa \texttt{G} od kojih postoji grana do čvora \texttt{v}.

Dovoljno je da jednom petljom prođemo sve čvorove grafa \texttt{G} i sve koji sadrže čvor \texttt{v} u svojoj listi dodamo u listu.

\lstinputlisting[language=Python, frame=single]{nedelje/4/kodovi/2.4.Incoming.py}

\subsection{Outgoing}
\label{outgoing}

Funkcija ima dva parametra, graf \texttt{G} i čvor \texttt{v}. Funkcija vraća listu čvorova do kojih postoje grane iz čvora \texttt{v} u grafu \texttt{G}, odnosno, vraća listu čvora \texttt{v}.

\lstinputlisting[language=Python, frame=single]{nedelje/4/kodovi/2.5.Outgoing.py}


\subsection{Bypass}
\label{bypass}

Funkcija ima četiri parametra, graf \texttt{G} i čvorove \texttt{u}, \texttt{v} i \texttt{w}. Povratna vrednost je novi graf sa izmenjenim granama. Još jednom, treba izbaciti grane $(u, v)$ i $(v, w)$, dodati novi čvor x (u kodu je \texttt{v'}, kako bi implementacija funkcije za pravljenje niske bila olakšana) povezati ga sa čvorom \texttt{u} i čvorom \texttt{w}.

Prvo, kopiramo graf \texttt{G} u \texttt{G\_p}, koji ćemo dalje menjati. Zatim, iz liste čvora \texttt{u} brišemo čvor \texttt{v}, a iz liste čvora \texttt{v} brišemo čvor \texttt{w}. Onda dodajemo novi čvor, \texttt{v'} u listu čvora \texttt{u}, a lista čvora \texttt{v'}  sadržaće samo čvor \texttt{w}. Na kraju vraćamo izmenjeni graf.


\lstinputlisting[language=Python, frame=single]{nedelje/4/kodovi/2.3.Bypass.py}


\subsection{Is Connected}
\label{isConnected}

Funkcija ima jedan parametar, graf \texttt{G}. Povratna vrednost je tipa boolean.

Održavamo mapu posećenosti. Za svaki čvor \texttt{v} u grafu \texttt{G} pozivamo pomoćnu proceduru koja će izvršiti obilazak grafa u dubinu (\texttt{DFS}) počevši iz čvora \texttt{v} nakon čega prekidamo petlju.

Za svaki čvor \texttt{v}u grafu \texttt{G} proveravamo da li se nalazi u mapi posećenih čvorova. Ukoliko bar jedan čvor nije u mapi, graf nije povezan i vraćamo \texttt{False}. Inače, vraćamo \texttt{True}.

\newpage
\lstinputlisting[language=Python, frame=single]{nedelje/4/kodovi/2.1.IsConnected.py}


\subsection{Create String From Path}
\label{createStringFromPath}

Funkcija ima jedan parametar, \texttt{path} - putanja na osnovu koje pravimo nisku.

Niska na početku sadrži karaktere ulaznog čvora prve grane. Ostale karaktere dobijamo proslaskom kroz sve grane na putanji, i izdvajanjem poslednjeg karaktera izlaznog čvora te grane.


\lstinputlisting[language=Python, frame=single]{nedelje/4/kodovi/2.6.CreateStringFromPath.py}


\subsection{DFS}
\label{DFS}

Funkcija ima tri parametra, graf \texttt{G}, čvor \texttt{v} i mapu posećenosti \texttt{visited}. Nema povratne vrednosti.

Čvor \texttt{v}, iz kog kreće obilazak grafa, obeležimo da je posećen. Zatim, prolazimo sve čvorove \texttt{w}, koji se nalaze u listi čvora \texttt{v}. Za svaki čvor koji do tad nije posećen, pozivamo istu proceduru.

\lstinputlisting[language=Python, frame=single]{nedelje/4/kodovi/2.2.DFS.py}



\subsubsection{Test primer}

\noindent\texttt{G} = \{'AT' : ['TC'], 'TC' : ['CG'], 'CG': ['GA','GG'], 'GA':['AT','AC'],  'AC':['CG'], 'GG':['GA']\}
\\\texttt{cycles} = ['ATCGACGGAT', 'TCGACGGATC', 'CGGATCGACG', 'GACGGATCGA', 'ACGGATCGAC', 'GGATCGACGG', 'CGACGGATCG', 'GATCGACGGA', 'ATCGGACGAT', 'TCGGACGATC', 'CGATCGGACG', 'GACGATCGGA', 'ACGATCGGAC', 'GGACGATCGG', 'CGGACGATCG', 'GATCGGACGA']

\section{String Spelled By Gapped Patterns}
\label{stringSpelledByGappedPatterns}

Funkcija ima tri parametra, \texttt{gapped\_patterns} - skup parova k-grama, \texttt{k} i \texttt{d} - rastojanje dva susedna para k-grama. Povratna vrednost je niska sačinjena od zadatih k-grama.

Razdvajamo parove u posebne liste, \texttt{first\_patterns} i \texttt{second\_patterns}. Zatim, formiramo prefiksne (od \texttt{first\_patterns}) i sufiksne (od \texttt{second\_patterns}) niske (funkcijom \texttt{string\_spe-\\lled\_by\_patterns}, \ref{stringSpelledByPatterns}).


Prefiksna i sufiksna niska imaju prefiks, odnoso, sufiks dužine \texttt{k+d}, a ostatak niski treba da im se poklopi. Tako da, prolazimo ove niske karakter po karakter. Prefiksu prolazimo počevši od pozicije \texttt{k+d} do kraja, a sufiksnu prolazimo od 0 do \texttt{len(prefiks\_string)-k-d}. Ukoliko naiđemo na nepoklapanje nukleotida na nekoj poziciji, ispisujemo odgovarajuću poruku i vraćamo praznu nisku. Ako su svi nukleotidi u redu, vraćamo nisku koja se dobije nadovezivanjem sufiksa iz sufiksne niske na prefiksnu nisku.



\lstinputlisting[language=Python, frame=single]{nedelje/4/kodovi/3.StringSpelledByGappedPatterns.py}

\subsection{String Spelled By Patterns}
\label{stringSpelledByPatterns}

Funkcija ima dva parametra, \texttt{patterns} - niske koje treba spojiti, i \texttt{k} - dužina svake niske. Povratna vrednost je DNK niska sastavljena od datih k-grama.

Početna vrednost niske \texttt{dna\_string} je k-gram bez poslednjeg karaktera. Prolaskom svih k-grama iz liste \texttt{patterns}, gradimo nisku nadovezivanjem poslednjeg karaktera iz svakog.


\lstinputlisting[language=Python, frame=single]{nedelje/4/kodovi/3.1.StringSpelledByPatterns.py}

\subsubsection{Test primer}

\noindent\texttt{gapped\_patterns} = [('CTG','CTG'),('TGA','TGA'),('GAC','GAC'),('ACT','ACT')]
\\\texttt{k} = 3
\\\texttt{d} = 1
\\\texttt{string} = CTGACTGACT
