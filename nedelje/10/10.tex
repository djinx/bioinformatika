\section{Trie Matching}

Funkcija ima dva parametra, \texttt{text} i sekvence koje tražimo u tekstu u drvenolikoj strukturi - \texttt{Trie}. Funkcija vraća listu sekvenci koje se nalaze u tekstu.

Krećemo od prazne liste pronađenih sekvenci. Petlju vrtimo dok nismo izbacili sve karaktere iz teksta. Tražimo sekvence koje se nalaze u trenutnom tekstu (funkcijom \texttt{prefixTriePatternMa-\\tching}, \ref{prefixTrieMatching}). Ukoliko smo pronašli neku, dodajemo povratnu vrednost u listu. Uklanjamo prvi karakter iz teksta i ponavljamo postupak.

\lstinputlisting[language=Python, frame=single]{nedelje/10/kodovi/1.TrieMatching.py}


\subsection{Prefix Trie Pattern Matching}
\label{prefixTrieMatching}

Funkcija ima dva parametra, \texttt{text} i sekvence koje tražimo u tekstu u drvenolikoj strukturi - \texttt{Trie}. Funkcija vraća sekvencu iz \texttt{Trie} koja se nalazi u tekstu.

Čvor \texttt{v} na početku je koreni i dodeljujemo mu vrednost 'root'. Prolazimo redom sve karaktere teksta. Ukoliko se neki od karaktera ne nalazi u listi čvora \texttt{v} vraćamo \texttt{False}. Inače, \texttt{v} dobija novu vrednost, i to onu koja odgovara karakteru \texttt{c} u listi čvora \texttt{v}. 

Zatim, proveravamo da li se znak '\$' nalazi u listi čvora \texttt{v}. Ako se nalazi, vraćamo vrednost koja odgovara karakteru '\$' u listi čvora \texttt{v}. Ako smo obradili sve karaktere teksta, znači da nismo naišli na poklapanje i vraćamo \texttt{False}.

\lstinputlisting[language=Python, frame=single]{nedelje/10/kodovi/1.1.PrefixTriePatternMatching.py}


\subsection{Trie Construction}
\label{trieConstruction}

Funkcija ima jedan parametar, listu \texttt{patterns} od koje treba napraviti \texttt{Trie}, što je povratna vrednost.

\texttt{Trie} će biti predstavljeno mapom čiji su ključevi nazivi čvorova (neke niske), a vrednosti su mape.

Inicijalizujemo \texttt{Trie} na praznu mapu, a onda dodajemo čvor sa nazivom \textit{'root'} čija je mapa takođe prazna na početku. Dodatno, pratimo broj čvorova, na početku to je 1.

Prolazimo redom sve sekvence liste \texttt{patterns} i dodajemo ih u \texttt{Trie} (funkcijom \texttt{add\_to\_trie}, \ref{addToTrie}). Obratiti pažnju da treba proslediti sekvencu koja sadrži znak '\$' na kraju! Funkcija vraća izmenjeni \texttt{Trie} i broj čvorova.


\lstinputlisting[language=Python, frame=single]{nedelje/10/kodovi/1.2.TrieConstruction.py}


\subsection{Add To Trie}
\label{addToTrie}

Funkcija ima četiri parametra, \texttt{Trie}, \texttt{pattern}, \texttt{number\_of\_nodes} i \texttt{pattern\_id}. Funkcija vraća izmenjeni \texttt{Trie} i broj čvorova.

Čvor sa oznakom 'root' smeštamo u promenljivu \texttt{current\_node}. Sekvencu prolazimo karakter po karakter, svaki ćemo smestiti u promenljivu \texttt{c}. Ukoliko se taj karakter nalazi u mapi trenutnog čvora, onda trenutni čvor postaje vrednost koja odgovara karakteru \texttt{c} iz mape čvora \texttt{current\_node}. U suprotnom, ukoliko nismo došli do kraja sekvence, odnosno, ukoliko je \texttt{c} različit od '\$' onda pravimo novi čvor.

Pravimo čvor sa oznakom \texttt{'i'} na koju nadovezujemo trenutni broj čvorova. Njegovu mapu postavljamo na praznu. Vrednost koja odgovara karakteru \texttt{c} iz mape čvora \texttt{current\_node} postavljamo na oznaku novog čvora. Trenutni čvor dobija vrednost novog čvora i uvećavamo broj čvorova za jedan.

Ukoliko smo došli do kraja sekvence, pravimo novi čvor sa oznakom \texttt{pattern\_id} i postavljamo ga za vrednost koja odgovara karakteru \texttt{c} iz mape čvora \texttt{current\_node}, a mapa novog čvora inicijalizovana je na praznu.


\lstinputlisting[language=Python, frame=single]{nedelje/10/kodovi/1.3.AddToTrie.py}

\subsubsection{Test primer}

\noindent\texttt{patterns} = ['ananas', 'and', 'antenna', 'banana', 'bandana', 'nab', 'nana', 'pan']
\\\texttt{query} = 'bananananaspand'
\\\texttt{Trie = trie\_construction(patterns)} 
\\\texttt{found\_patterns} = [3, 6, 6, 6, 7, 1] 


\section{Pattern Matching With Suffix Array}

Funkcija ima dva parametra, \texttt{suffix\_array} - sufiksni niz napravljen na osnovu teksta, i \texttt{pattern} - sekvenca koju tražimo u tekstu. Povratna vrednost je lista pozicija na kojima počinje sekvenca u tekstu.

Održavamo vrednosti \texttt{top}, koja kreće od pozicija 0, i \texttt{bottom}, koja kreće od pozicije |suffix\_array|-1. Dok važi da je \texttt{top} manji od \texttt{bottom} izračunavamo srednji element \texttt{mid}, kao srednju vrednost ove dve. 

Ukoliko je dužina sufiksa središnjeg veća od dužine sekvence koju tražimo, onda proveravamo i da li je prefiks dužine |pattern| u središnjem sufiksu jednak sekvenci \texttt{pattern}. Ako jeste, vraćamo povratnu vrednost funkcije \texttt{find\_neighborhood} (\ref{findNeighborhood}). 

Sledeći korak odnosi se na ažuriranje vrednosti za \texttt{top} i \texttt{bottom} kao u binarnoj pretrazi. Tako da, ako je sekvenca manja od središnjeg elementa sufiksnog niza, onda se \texttt{bottom} pomera na jedno mesto ispred \texttt{mid}, u suprotnom pomeramo \texttt{top} na mesto iza \texttt{mid}.

\lstinputlisting[language=Python, frame=single]{nedelje/10/kodovi/2.PatternMatchingWithSuffixArray.py}


\subsection{Find Neighborhood}
\label{findNeighborhood}


Funkcija ima tri parametra, \texttt{suffix\_array}, \texttt{mid} i \texttt{pattern}. Povratna vrednost je lista pozicija na kojima počinje sekvenca \texttt{pattern}.

Krećemo od središnjeg elementa i pomeramo se nadole i nagore dok su uspunjeni odgovarajući uslovi. 

Za kretanje naviše bitno je da brojač ostane pozitivna vrednost, da dužina gornjeg elementa sufiksnog niza bude veća od dužine sekvence koju tražimo i da je prefiks dužine |pattern| gornje niske u sufiksnom nizu jednak sekvenci \texttt{pattern}. 

Za kretanje naniže uslovi su da brojač ne pređe dužinu niza, da je dužina donjeg elementa sufiksnog niza veća od dužine sekvence i da je prefiks dužine |pattern| donje niske u sufiksnom nizu jednaka sekvenci \texttt{pattern}.

Kada su obe granice određene, postavljamo ih u petlji. Krećemo od gornje granice pomerene za 1 i idemo do donje. U listu pozicija (koja je inicijalno prazna) dodajemo razliku dužine sufiksnog niza i dužine elementa na poziciji \texttt{i}.

Pre nego što vratimo listu, sortiramo je rastuće.

\lstinputlisting[language=Python, frame=single]{nedelje/10/kodovi/2.1.FindNeighborhood.py}


\subsection{Suffix Array Construction}
\label{suffixArrayConstruction}

Funkcija ima jedan parametar, \texttt{string}. Povratna vrednost je sufiksni niz napravljen na osnovu niske \texttt{string}.

Sufiksni niz na početku je prazna lista, a niski nadovezujemo '\$' na kraj. Nisku prolazimo pomoću indeksa \texttt{i}, a u svakoj iteraciji u sufiksni niz dodajemo podnisku koja počinje na poziciji \texttt{i}. Niz sortiramo i vraćamo iz funkcije.

\lstinputlisting[language=Python, frame=single]{nedelje/10/kodovi/2.2.SuffixArrayConstruction.py}


\subsubsection{Test primer}

\texttt{string} = 'ananas'
\\\texttt{suffix\_array = suffix\_array\_construction(string)} 
\\\texttt{pattern} = 'nana'
\\\texttt{positions} = [1]


U nastavku su opisane modifikacije ovih funkcija tako da rade kada umesto jedne niske imamo niz niski.

\section{Pattern Matching With Suffix Array (multiple)}

Funkcija ima dva parametra, \texttt{suffix\_array} - sufiksni niz napravljen na osnovu teksta, i \texttt{pattern} - sekvenca koju tražimo u tekstu. Povratna vrednost je lista pozicija na kojima počinje sekvenca u tekstu.

Obratiti pažnju da sufikni niz više nije niz niski već niz torki. Na prvom mestu nalazi se sufiks, na drugom je redni broj niske čiji je to sufiks, a na trećem je indeks na kom počinje sufiks u niski.

Održavamo vrednosti \texttt{top}, koja kreće od pozicija 0, i \texttt{bottom}, koja kreće od pozicije |suffix\_array|-1. Dok važi da je \texttt{top} manji od \texttt{bottom} izračunavamo srednji element \texttt{mid}, kao srednju vrednost ove dve. 

Ukoliko je dužina sufiksa središnjeg veća od dužine sekvence koju tražimo, onda proveravamo i da li je prefiks dužine |pattern| u središnjem sufiksu jednak sekvenci \texttt{pattern}. Ako jeste, vraćamo povratnu vrednost funkcije \texttt{find\_neighborhood} (\ref{findNeighborhoodM}). 

Sledeći korak odnosi se na ažuriranje vrednosti za \texttt{top} i \texttt{bottom} kao u binarnoj pretrazi. Tako da, ako je sekvenca manja od središnjeg elementa sufiksnog niza, onda se \texttt{bottom} pomera na jedno mesto ispred \texttt{mid}, u suprotnom pomeramo \texttt{top} na mestoiza \texttt{mid}.

\lstinputlisting[language=Python, frame=single]{nedelje/10/kodovi/3.PatternMatchingWithSuffixArrayM.py}


\subsection{Find Neighborhood (multiple)}
\label{findNeighborhoodM}


Funkcija ima tri parametra, \texttt{suffix\_array}, \texttt{mid} i \texttt{pattern}. Povratna vrednost je lista pozicija na kojima počinje sekvenca \texttt{pattern}.

Krećemo od središnjeg elementa i pomeramo se odozgo nadole i odozdo nagore dok su uspunjeni odgovarajući uslovi. 

Za kretanje odozgo naniže bitno je da brojač ostane pozitivna vrednost, da dužina niske gornjeg elementa sufiksnog niza bude veća od dužine sekvence koju tražimo i da je prefiks dužine |pattern| gornje niske u sufiksnom nizu jednaka sekvenci \texttt{pattern}. 

Za kretanje odozdo naviše uslovi su da brojač ne pređe dužinu niza, da je dužina niske donjeg elementa sufiksnog niza veća od dužine sekvence i da je prefiks dužine |pattern| donje niske u sufiksnom nizu jednaka sekvenci \texttt{pattern}.

Kada su obe granice određene, postavljamo ih u petlji. Krećemo od gornje granice pomerene za 1 i idemo do donje. U listu pozicija (koja je inicijalno prazna) dodajemo par indeksa - redni broj niske (tj. drugi element torke elementa sufiksnog niza) i indeks na kom počinje sufiks (tj. treći element torke elementa sufiksnog niza) elementa na poziciji \texttt{i}.

Pre nego što vratimo listu, sortiramo je rastuće.

\newpage
\lstinputlisting[language=Python, frame=single]{nedelje/10/kodovi/3.1.FindNeighborhoodM.py}


\subsection{Suffix Array Construction (multiple)}
\label{suffixArrayConstructionM}

Funkcija ima jedan parametar, \texttt{strings}. Povratna vrdnost je sufiksni niz napravljen na osnovu niza niski \texttt{strings}.

Sufiksni niz na početku je prazna lista, a gradićemo ga tako da elementi budu torke, a ne sami sufiksi. Jedna torka sadrži sufiks jedne niske iz niza, redni broj te niske i poziciju na kojoj počinje sufiks u niski.

Redom indeksiramo elemente niza (indeksom \texttt{s}), i svakom na kraj nadovezujemo '\$'. Zatim, nisku prolazimo pomoću indeksa \texttt{i}, a u svakoj iteraciji u sufiksni niz dodajemo torku koja sadrži podnisku niske koja počinje na poziciji \texttt{i}, poziciju niske u nizu tj. \texttt{s} i indeks na kom počinje sufiks, odnosno, \texttt{i} . Niz sortiramo i vraćamo iz funkcije.

\lstinputlisting[language=Python, frame=single]{nedelje/10/kodovi/3.2.SuffixArrayConstructionM.py}



\subsubsection{Test primer}

\noindent\texttt{strings} = ['ananas', 'and', 'antenna', 'banana', 'bandana', 'nab', 'nana', 'pan']
\\\texttt{suffix\_array = suffix\_array\_construction(strings)}
\\\texttt{pattern} = 'an'
\\\texttt{positions} = [(0, 0), (0, 2), (1, 0), (2, 0), (3, 1), (3, 3), (4, 1), (4, 4), (6, 1), (7, 1)]


\section{BW Matching}

Funkcija ima tri parametra, \texttt{first\_column}, \texttt{last\_column} i \texttt{pattern}. Povratna vrednost je pozicija na kojoj počinje sekvenca \texttt{pattern} u tekstu.

Održavamo vrednosti \texttt{top}, koja kreće od pozicija 0, i \texttt{bottom}, koja kreće od pozicije |last\_column|-1. Petlja se vrti dok važi da je \texttt{top} manji od \texttt{bottom}. Ako sekvenca ima još karaktera, izdvajamo poslednji simbol u \texttt{symbol} a \texttt{pattern} skraćujemo za posledni karakter. Izdvajamo podskup poslednje kolone na intervalu \texttt{[top, bottom]} u \texttt{subset}. 

Metodom \texttt{index} proveravamo da li \texttt{subset} sadrži karakter \texttt{symbol} (funkcija vraća indeks, ako pronađe element, odnosno -1, ako ga ne nađe). 

Ukoliko ga sadrži, želimo da odredimo \texttt{top\_index} i \texttt{bottom\_index}, koje incijalizujemo na -1. Petljom prolazimo poslednju kolonu, od \texttt{top} do \texttt{bottom} + 1. Ukoliko je \texttt{symbol} jednak karakteru poslednje kolone, menjamo \texttt{top\_index} i \texttt{bottom\_index}. Gornji indeks menjamo samo ako već nije dobio neku vrednost, odnosno, ako i dalje ima vrednost -1. U tom slučaju ga postavljamo na vrednost indeksa petlje, a donji indeks u svakom slučaju dobija tu vrednost.

Sada želimo da izbrojimo koliko se karaktera u poslednjoj koloni poklapa sa \texttt{symbol}, do gornjeg indeksa (\texttt{top\_count}) i do donjeg indeksa (\texttt{bottom\_count}). To vršimo jednostavnim prolaskom poslednje kolone i upoređivanjem karaktera.

Treba još ažurirati \texttt{top} i \texttt{bottom}. Oba ažuriramo pomoću funkcije \texttt{last\_to\_first} (\ref{lastToFirst}), s tim da prosleđujemo različite elemente poslednje kolone (element na poziciji \texttt{top\_index}/\texttt{bottom\_index}) i brojače (\texttt{topCount}/\texttt{bottom\_count}). 

Ukoliko \texttt{subset} ipak ne sadrži \texttt{symbol}, vraćamo 0.

Ukoliko nema više karaktera u sekvenci \texttt{pattern}, vraćamo razliku \texttt{bottom} i \texttt{top} uvećanu za jedan.

\subsection{Last To First}
\label{lastToFirst}

Funkcija ima tri parametra, \texttt{first\_column}, karakter \texttt{c} i broj pojavljivanja tog karaktera u poslednjoj koloni \texttt{count}. Povratna vrednost je poslednji indeks na kom počinje taj karakter u prvoj koloni.

Dakle, prolazimo redom sve karaktere prve kolone. Ukoliko je \texttt{c} jednak karakteru prve kolone na poziciji \texttt{i} onda u zavisnosti od broja pojavljivanja radimo sledeće. Ako je broj pojavljivanja jednak 1, vraćamo indeks \texttt{i}. Inače, umanjujemo brojač za jedan.


\lstinputlisting[language=Python, frame=single]{nedelje/10/kodovi/4.1.LastToFirst.py}


\newpage 

\lstinputlisting[language=Python, frame=single]{nedelje/10/kodovi/4.BWMatching.py}

\newpage

\subsection{BWT}
\label{BWT}

Funkcija ima jedan parametar, nisku \texttt{s}. Povratna vrednost poslednja kolona matrice permutacija niske \texttt{s}.

Polazimo od prazne matrice, koja je predstavljena kao niz niski. Na kraj niske \texttt{s} nadovezujemo karakter '\$'. Petlja se izvršava \texttt{|s|} puta. U svakoj iteraciji, u matricu nadovezujemo nisku \texttt{s} a zatim je ciklično pomeramo za jedno mesto u levo. Nakon petlje, sortiramo matricu i vraćamo listu poslednjih karaktera iz svakog reda matrice, odnosno, poslednju kolonu.

	
\lstinputlisting[language=Python, frame=single]{nedelje/10/kodovi/4.2.BWT.py}


\subsubsection{Test primer}

\noindent\texttt{s} = 'panamabananas'
\\\texttt{last\_column = BWT(s)}
\\\texttt{pattern} = 'ana'
\\\texttt{first\_column = last\_column[:]}
\\\texttt{position} = 3 

